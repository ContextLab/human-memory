\documentclass[10pt]{article}
\usepackage{amsmath}
\usepackage{setspace}
\usepackage{pxfonts}
%\usepackage{graphicx}
\usepackage{geometry}


\geometry{letterpaper,left=.5in,right=.5in,top=0.5in,bottom=.75in,headsep=5pt,footskip=20pt}

\title{PSYC 51.09: Problem Set 4}
%\author{Jeremy R. Manning}
\date{}

\begin{document}
\maketitle
\vspace{-0.75in}
\section*{Introduction}
This problem set is intended to solidify the concepts you learned
about in this week’s lectures and readings.  \textit{After attempting each question
  on your own}, you are encouraged to work together with your classmates in small groups, and/or to post and answer questions on the course’s Canvas site.

You must upload your answers before the due date in order to receive credit.  No late submissions will be accepted.

\section*{Readings}
\begin{enumerate}
\item Read Chapter 4 of \textit{Foundations of Human Memory}.  What were your thoughts on the reading?  For example, did you learn something interesting?  Were you surprised by something?  Do you disagree with the author?  Did you think some concept was described especially well (or confusingly)?  \textbf{(Ungraded)}
\end{enumerate}

\section*{Graded questions}

\begin{enumerate}
\item Your friend (a native English speaker) is learning Klingon, and
  he wants to learn some new vocabulary words for his upcoming midterm
  exam.  Unfortunately his midterm exam is in just a few hours, and he
  hasn't gotten around to studying quite yet.  When he learns that
  you're taking PSYC 51.09, he begs you for some tips to help him
  quickly learn the words.  What can you tell your friend to help him
  learn his vocabulary words quickly in time for his exam?  For
  example, suppose he's already made a set of flashcards where one
  side lists the word in English and the other lists the Klingon
  translation.  Should your friend study the flashcards in the same
  order each time, or shuffle the flashcards with each repetition?
  Should he ``peek'' at the answers he's unsure about?  Should he
  shuffle only the ``tricky'' words (e.g. that he couldn't translate)
  back into the deck with each round of repetitions, or should he go
  through the full set of flashcards each time?  When should he stop
  studying the words?  Provide four recommendations, along with clear
  explanations for why you are making those particular
  recommendations.  \textbf{(4(ish) paragraphs; 1 per recomendation.  Each recommendation must include a clear
    explanation of the recommended studying technique and the reason
    you are recommending it, and you should use 1--2 paragraphs to
    describe and explain each recommendation.)}

\item The recently developed synthetic lifeform \textit{Roboticus Metallicus} has a
  perfect memory for everything it experiences.  It also has no way of
  prioritizing one memory over another, or of strengthening one
  association versus another.  Suppose a member of the species signs
  up to do the free association experiment you're running for your
  senior project.
\begin{enumerate}
\item Would you expect the \textit{Roboticus Metallicus}'s free
  association responses to differ from those of a typical \textit{Homo
    Sapiens}?  If so, how?  If not, why not?  \textbf{(2 paragraphs)}
\item Suppose you run many \textit{Roboticus Metallici} in your
  experiment-- enough to build up a large and reliable database of free
  associate responses.  Next you decide to construct a ``thought space''
  based on their responses (analogous to the ``Word Association
  Space'' idea we discussed in class).  Describe some features of the
  \textit{Roboticus Metallicus}-based thought space (e.g. as compared
  with Word Association Spaces constructed using human data).
  What would the thought space ``look'' like?  How would words be
  arranged?
  \textbf{(1-2 paragraphs.)}
\end{enumerate}
\end{enumerate}

\end{document}


