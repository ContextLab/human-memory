\documentclass[11pt]{article}
\usepackage{amsmath}
\usepackage{setspace}
\usepackage{pxfonts}
%\usepackage{graphicx}
\usepackage{geometry}


\geometry{letterpaper,left=.5in,right=.5in,top=0.5in,bottom=.75in,headsep=5pt,footskip=20pt}

\title{PSYC 51.09: Problem Set 7}
%\author{Jeremy R. Manning}
\date{}

\begin{document}
\maketitle
\vspace{-0.75in}
\section*{Introduction}
This problem set is intended to solidify the concepts you learned
about in this week's lectures and readings.  Your responses will be
worth 5\% of your final grade.  \textit{After attempting to answer
  each question on your own}, you are encouraged to work together with your classmates in small groups, and/or to post and answer questions on the course’s Canvas site.  \textbf{\textit{However, you must clearly indicate who your collaborated with and submit your own (uniquely worded) responses.}}

You must upload your problem set before \textbf{Friday, November 22, 2019 at 11:59 pm}.  No late submissions will be accepted.

\section*{Readings and ungraded questions}
\begin{enumerate}
\item Read Chapter 8 of \textit{Foundations of Human Memory}.  What were your thoughts on the reading?
  \textbf{(Ungraded)}

\item Read Chapter 9 of \textit{Foundations of Human Memory}.  What were your thoughts on the reading?
  \textbf{(Ungraded)}

\item Congratulations on finishing \textit{Foundations of Human
    Memory}!  What were your overall thoughts about the textbook?
  Which were your favorite parts?  Which parts were least clear?  Are
  there topics you were hoping the textbook had covered (or gone into
  more depth about)?  \textbf{(Ungraded)}

\item Optional: read Baldassano et al.\ (2016) [File name:
  \texttt{BaldEtal16.pdf}].  What
were your thoughts on the reading? \textbf{(Ungraded)}
\end{enumerate}

\section*{Graded questions}
In answering the questions below, consider this week's material in the
context of the other material we've learned throughout the course.

\begin{enumerate}
\item How do our brains organize and spontaneously retrieve memories?
  Use an example if it helps, or you can give a general answer.
  \textbf{(2-3 paragraphs, 2 points)}

\item In your view, what is the single greatest challenge to our understanding
  of human memory?  For example, where is our knowledge of ``how
  memory works'' weakest?  Or, what sorts of questions about memory are
  the most difficult to study?  Why?  \textbf{(2-3 paragraphs, 2 points)}

\item What would need to happen in order to overcome (solve) the challenge(s) you
  identified above?  Do you think it's possible and/or will ever be
  possible to address that challenge or are we doomed to always have
  an incomplete understanding of memory?  Why?  \textbf{(2-3 paragraphs, 1 point)}
\end{enumerate}

\end{document}


