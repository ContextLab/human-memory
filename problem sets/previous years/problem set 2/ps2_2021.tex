\documentclass[11pt]{article}
\usepackage{amsmath}
\usepackage{setspace}
\usepackage{pxfonts}
\usepackage{geometry}
\usepackage{graphicx}
\usepackage[table, xcdraw]{xcolor}


\geometry{letterpaper,left=.5in,right=.5in,top=0.5in,bottom=.75in,headsep=5pt,footskip=20pt}

\title{PSYC 51.09: Problem Set 2}
%\author{Jeremy R. Manning}
\date{}

\begin{document}
\maketitle
\vspace{-0.75in}
\section*{Overview}
This problem set is intended to solidify the concepts you learned
about in this week's lectures and readings.  \textit{After
  attempting each question on your own}, you are encouraged to work
together with your classmates in small groups, and/or to post and
answer questions on the course’s Canvas site.  \textbf{\textit{You
    must clearly indicate who your collaborated with and submit your
    own (uniquely worded) responses.}}

Please upload your problem set to Canvas (as a Word or PDF file)
before the due date.  No late submissions will be accepted.

\section*{Readings and ungraded questions}
\begin{enumerate}
\item Read Chapter 2 of \textit{Foundations of Human Memory}.  What were your thoughts on the reading?
  \textbf{(Ungraded)}
\end{enumerate}

\section*{Graded questions}
\begin{enumerate}
\item Suppose the table below contains data you've collected from one
  participant in a recognition memory experiment.  They were tested
  with 20 items (TRIAL) which included a mix of targets and lures
  (STATUS).  For each item, they made a 7-point CONFIDENCE judgement:
  1 $=$ sure it was not on the list; 7 $=$ sure it was on the list.
\begin{enumerate}
  \item Plot (by hand) the ROC curve for this participant.  Be sure to label
    axes and put numbers on the axes.  Show your work!
  \item Draw (and label) a dotted line on the ROC curve to indicate what
    it would look like for a participant who mixed up the instructions
    and reversed the ratings scale in their responses (i.e., they
    responded 1 if they were sure the item was \textit{old} and 7 if
    they were sure the item was \textit{new}).
  \item Draw another (labeled) dotted line on the ROC curve to
    indicate what it would look like for a participant who responded
    normally, except that they rounded all of their 
    responses to 7 if they internally judged their confidence at 5 or
    higher (in other words, draw the ROC curve when responses of 5, 6,
    or 7 below are all replaced with 7)
\end{enumerate}

\begin{table}[h]
\small
\centering
\begin{tabular}{|>{\columncolor[HTML]{d1d3d4}}r |
  c|c|c|c|c|c|c|c|c|c|c|c|c|c|c|c|c|c|c|c|}
\hline
{\color[HTML]{000000} TRIAL}      & 1 & 2 & 3 & 4 & 5 & 6 & 7 & 8 & 9
  & 10 & 11 & 12 & 13 & 14 & 15 & 16 & 17 & 18 & 19 & 20\\\hline
{\color[HTML]{000000} STATUS}     & T & L & T & T & T & L & L & L & T & L & L & T & L & L & T & T & L & L & T & T\\\hline
{\color[HTML]{000000} CONFIDENCE} & 4 & 5 & 2 & 6 & 6 & 4 & 6 & 3 & 6 & 2 & 4 & 5 & 2 & 1 & 3 & 4 & 6 & 1 & 7 & 2\\\hline
\end{tabular}
\end{table}

\item Consider some aspect of recognition memory that neither
  strength-based models (e.g., strength theory, the Yonelinas
  familiarity-recollection model, and the variable-recollection model)
  nor scanning models (e.g., serial self-terminating scan, serial
  exhaustive scan, parallel search models) can explain.  Outline some
  ideas for extending (or combining) one or more of these models in a
  way that could help to account for that phenomenon.   (\textbf{3
    paragraphs})

\end{enumerate}
\end{document}


