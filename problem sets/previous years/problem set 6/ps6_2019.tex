\documentclass[11pt]{article}
\usepackage{amsmath}
\usepackage{setspace}
\usepackage{pxfonts}
%\usepackage{graphicx}
\usepackage{geometry}


\geometry{letterpaper,left=.5in,right=.5in,top=0.5in,bottom=.75in,headsep=5pt,footskip=20pt}

\title{PSYC 51.09: Problem Set 6}
%\author{Jeremy R. Manning}
\date{}

\begin{document}
\maketitle
\vspace{-0.75in}
\section*{Introduction}
This problem set is intended to solidify the concepts you learned
about in this week's lectures and readings.  Your responses will be
worth 5\% of your final grade (note: the problem set totals 3 points,
but this will be corrected to 5 percentage points of your final
grade).  \textit{After attempting each problem on your own}, you are
encouraged to work together with your classmates in small groups,
and/or to post and answer questions on the course’s Canvas site.
\textbf{\textit{However, you must clearly indicate who your
    collaborated with and submit your own (uniquely worded)
    responses.}}

We will go over the answers to this problem set in class on
\textbf{Thursday, November 14, 2019 at 10:10 am}.  You must upload your answers before then in order to receive credit.  No late submissions will be accepted.

\section*{Readings and ungraded questions}
\begin{enumerate}
\item Read Chapter 6 of \textit{Foundations of Human Memory}.  What were your thoughts on the reading?
  \textbf{(Ungraded)}

\item Read Chapter 7 of \textit{Foundations of Human Memory}.  What were your thoughts on the reading?
  \textbf{(Ungraded)}

\item Read the chapter \textit{Context Reinstatement} (Mann19.pdf).  What were your thoughts on the reading?
  \textbf{(Ungraded)}

\item Optionally, read the chapter \textit{The Role of Context in
    Episodic Memory} in \textit{The Cognitive Neurosciences}
  (MannEtal15.pdf).  What were your thoughts on the reading?
  \textbf{(Ungraded)}


\item Brainstorm possible topics for your final paper.  Write down 2-3
  of the most promising options.  If you will be working with a group,
  also specify the 1-2 students you'll be working with. If you haven't
  already done so, please sign up for a time to meet with me to
  discuss your final paper on context-lab.youcanbook.me.\textbf{(Ungraded)}
\end{enumerate}

\section*{Graded questions}
\begin{enumerate}
\item Suddenly it comes to you in the middle of the night: you've figured
  out how memory works!  You furiously scribble down as many details
  as you can, before the thoughts are lost forever.  You have
  created...\textit{\textbf{the SHAM model}}.

The next morning, you look back at your scribbles, and you realize
that the SHAM model is actually very similar to another model you've studied in
PSYC 51.09-- the SAM model.  In fact, it's the identical in every way,
except for one ``small'' detail.  In the SAM model, the associations
between items that occupy the short term memory store at the same time
are strengthened.  But in the SHAM model, the opposite happens.  Specifically,
items that occupy the short term memory store at the same time become
\textit{less strongly associated}.  (Similarly, items in the short
term memory store become \textit{less} associated with context,
according to the SHAM model.)

With a sinking feeling, you realize that the SHAM model is going to
make some pretty strange predictions about people's free recall
behaviors.
\begin{enumerate}
\item What would the SHAM model predict the \textit{serial position
    curves} (recall probability by presentation position) will look like for immediate free recall and
  delayed free recall?  Draw two curves (and label which is which).
  Explain (in 1 paragraph) why you drew the curves the way you
  did.  \textbf{(1 point)}

\item What does the SHAM model predict people's temporal clustering
  patterns (Fig. 7.4) will look like (for immediate free recall)?
  Draw a curve analogous to Fig. 7.4, and explain (in 1 paragraph) why
  you drew the curve the way you did.  \textbf{(1 point)}
\end{enumerate}

\item Since graduating from Dartmouth last year, \textit{future you} has been
  pouring all of your energy into a new tech startup you founded with
  some of your Dartmouth classmates.  You've created a new search
  engine, \textit{SHAM2} (no relation to that strange model you
  thought up back when you were an undergrad) that uses the Temporal
  Context Model to more effectively help people to retrieve personal
  files on their computer.  For example, suppose you're listening to a song
  while working on a problem set for PSYC 51.09.  Those two co-active events will become
  linked in a ``context database'' so that, later, when you search for
  the song it will also bring up the document, or when you search for the
  problem set (or its contents) it will bring up the song.

  You've been pitching your search engine to a friend-of-a-friend who
  knows a group of investors who might be interested.  Your contact
  arranges a meeting, but the investors make it clear that they
  are going to be deciding between backing your TCM-based search
  engine, or another search engine that has no contextual awareness
  but is slightly faster than the one you've been working on.

You need to make your case: why should they back your company?
Explain to the investors why they should care about context-aware
search and invest their millions (billions?) in your company instead of your
competition.  \textbf{(2-3 paragraphs; 1 point)}

\end{enumerate}

\end{document}


