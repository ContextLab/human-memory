\documentclass[11pt]{article}
\usepackage{amsmath}
\usepackage{setspace}
\usepackage{pxfonts}
%\usepackage{graphicx}
\usepackage{geometry}


\geometry{letterpaper,left=.5in,right=.5in,top=0.5in,bottom=.75in,headsep=5pt,footskip=20pt}

\title{PSYC 51.09: Problem Set 1}
%\author{Jeremy R. Manning}
\date{}

\begin{document}
\maketitle
\vspace{-0.75in}
\section*{Overview}
This problem set is intended to solidify the concepts you learned
about in this week's lectures and readings.  \textit{After attempting
each question on your own}, you are encouraged
to work together with your classmates in small groups, and/or to post
and answer questions on the course’s Canvas site.

Please upload your problem set to Canvas (as a Word or PDF file)
before the due date.  No late submissions will be accepted.

\section*{Readings and ungraded questions}
\begin{enumerate}
\item Read Chapter 1 of \textit{Foundations of Human Memory}.  What were your thoughts on the reading?
  \textbf{(Ungraded)}

\item Optional: read Tulving (1972)'s highly influential book chapter
  on Episodic and Semantic memory
  \textbf{(Ungraded)}

\item Optional: watch the movie \underline{Memento} (2000).  \textbf{(Ungraded)}
\end{enumerate}

\section*{Graded questions}
\begin{enumerate}

\item Describe one aspect of memory that you are absolutely certain is
  true.  Explain how you know-- be as specific as possible.
  \textbf{(2-3 paragraphs)}

\item How might you study each the following types of
  memory?  Describe (briefly!) an experiment for each.  Also describe
  what you think your proposed experiment can and can't tell you about that
  type of memory.
  \begin{enumerate}
  \item Memory for autobiographical events (e.g., what did you do last Tuesday?)
    \item Memory for spatial locations (e.g., where you put your cell
      phone?)
      \item Memory for facts (e.g., when was Dartmouth founded?)
      \end{enumerate}
\textbf{(1-2 paragraphs per type of memory, clearly delineated under
  different headings)}

\item Life on the planet \textit{Thear} is nearly identical to life on
  planet Earth, but with one key difference.  Notably, ionizing
  radiation from the nearby binary star system interacts with a type
  of dense atmospheric quantum fog found only on Thear.  These quantum
  interactions occasionally disrupt memory.  In particular, each
  morning at the precise start of sunrise, all memories from one day
  (selected at random) from each person's past are heavily modified so
  as to become completely unrecognizable from the original memories.
  Thearians are oblivious to this process as it is happening-- the
  modified memories still ``feel'' just as real as unmodified
  memories, and Thearians don't know which memory might have been
  altered on any given day.  Could human Thearian society still
  function?  How might you expect the human experience on Thear to
  differ from the human experience on Earth?  Explain your reasoning
  and state any assumptions.  \textbf{(2-3 paragraphs)}
\end{enumerate}

\end{document}
