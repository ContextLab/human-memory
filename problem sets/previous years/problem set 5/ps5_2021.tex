\documentclass[11pt]{article}
\usepackage{amsmath}
\usepackage{setspace}
\usepackage{pxfonts}
%\usepackage{graphicx}
\usepackage{geometry}


\geometry{letterpaper,left=.5in,right=.5in,top=0.5in,bottom=.75in,headsep=5pt,footskip=20pt}

\title{PSYC 51.09: Problem Set 5}
%\author{Jeremy R. Manning}
\date{}

\begin{document}
\maketitle
\vspace{-0.75in}
\section*{Introduction}
This problem set is intended to solidify the concepts you learned
about in this week's lectures and readings.  \textbf{After attempting each problem
  on your own,} you are encouraged to work together with your classmates in small groups, and/or to post and answer questions on the course’s Canvas site.

You must upload your answers before the due date in order to receive credit.  No late submissions will be accepted.

\section*{Readings}
\begin{enumerate}
\item Read Chapter 5 of \textit{Foundations of Human Memory} (if you
  have not already done so).  What were your thoughts on the reading?
  \textbf{(Ungraded)}

\item \textit{Optional.}  If you'd like to learn about \textit{deep neural networks} (an
  extension of the Hopfield networks we learned about in class and in
  Chapter 5) watch this YouTube video:
  \texttt{https://tinyurl.com/kvbw872}.  What'd you think?
  \textbf{(Ungraded)}

\item \textit{Optional.} If you'd like to learn about how network
  patterns in our brains reflect our ongoing thoughts, read Owen et
  al. (2021).  Thoughts?  \textbf{(Ungraded)}

\item \textit{Optional.} If you'd like to learn more about how we can
  intentionally forget, read Manning et al. (2016).  You can also
  listen to a radio segment on the study here:
  \texttt{https://tinyurl.com/y25fwklm}. \textbf{(Ungraded)}

  \item \textit{Optional.} Sievers and Momennejad (2019) propose an
    approach for ``deleting'' specific targeted  memories by
    presenting tailored sequences of stimuli.  Can you think of any
    interesting applications and/or implications of this work?  \textbf{(Ungraded)}
\end{enumerate}

\section*{Graded questions}

For this problem set, your job is to create your own neural
  network model of memory (a Hopfield network).  Below are two
  memories, $\mathbf{m}_1$ and $\mathbf{m}_2$ that you will store in
  your network.  Use the techniques we discussed in class (and in
  the book), along with the provided equations, to answer the
  following questions.  Show your work!
\[
  \mathbf{m}_1=
  \begin{pmatrix}
    1\\
    -1\\
    -1\\
    1\\
    -1\\
    -1\\
  \end{pmatrix}\quad
  \mathbf{m}_2=
  \begin{pmatrix}
    -1\\
    -1\\
    1\\
    -1\\
    -1\\
    1\\
  \end{pmatrix}\quad
\mathbf{x}_1=
  \begin{pmatrix}
    -1\\
    -1\\
    0\\
    0\\
    0\\
    0\\
  \end{pmatrix}\quad
\mathbf{x}_2=
  \begin{pmatrix}
    1\\
    -1\\
    0\\
    0\\
    0\\
    0\\
  \end{pmatrix}
\]
Learning rule:
\[
W(i,j) = \sum_{k = 1}^L a_k(i)a_k(j)
\]
Dynamic rule:
\[
a(i) = \mathrm{sign}\left(\sum_{j=1}^N W(i,j)a(j)\right)
\]

\begin{enumerate}
\item Create a weight matrix, using Hebbian learning, that contains
  both $\mathbf{m}_1$ and $\mathbf{m}_2$ as stable memories.

\item For each of the partial cues, $\mathbf{x}_1$ and $\mathbf{x}_2$,
  the activity of the first two neurons is known.  Use
  \textbf{asynchronous updating} to calculate the activities of the
  remaining four neurons (in whatever order you want).  Can the
  network retrieve both memories?  Hint: update neurons 3, 4, 5, and 6
  (in any order).  Then continue updating those 4 neurons until none of the
  values change to show that the network has stabilized.
\end{enumerate}

\end{document}


