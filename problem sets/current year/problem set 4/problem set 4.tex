\documentclass[10pt]{article}
\usepackage{amsmath}
\usepackage{setspace}
\usepackage{pxfonts}
%\usepackage{graphicx}
\usepackage{geometry}


\geometry{letterpaper,left=.5in,right=.5in,top=0.5in,bottom=.75in,headsep=5pt,footskip=20pt}

\title{PSYC 51.09: Problem Set 4}
%\author{Jeremy R. Manning}
\date{}

\begin{document}
\maketitle
\vspace{-0.75in}
\section*{Introduction}
This problem set is intended to solidify the concepts you learned about in this
week's lectures and readings. \textit{After attempting each question on your
own}, you are encouraged to work together with your classmates in small groups,
consult with ChatGPT or other tools, and/or to post and answer questions on the
course’s Canvas site.

Please upload your problem set to Canvas (as a Word or PDF file)
before the due date.  No late submissions will be accepted.

\section*{Readings}
\begin{enumerate}
\item Read Chapter 4 of \textit{Foundations of Human Memory}.  What were your thoughts on the reading?  For example, did you learn something interesting?  Were you surprised by something?  Do you disagree with the author?  Did you think some concept was described especially well (or confusingly)?  \textbf{(Ungraded)}

\item Optional: submit a multiple-choice question based on the materials
covered in this week's lectures, readings, and this problem set. You should
calibrate the difficulty so that 60--70\% of your classmates answer it
correctly on an exam.  If your question is chosen and you hit your target,
you will receive and extra credit point on that exam.  \textbf{(Ungraded)}
\end{enumerate}

\section*{Graded questions}

\begin{enumerate}

\item Your friend is studying for their Organic Chemistry midterm, and they
need to memorize the structures of the 20 amino acids. They've decided to use
flashcards to help them learn the structures. Sadly, after a long night of
thinking about optimal flashcard studying strategies, they've come down with a
bad case of the dreaded ``the exam is in just a few hours and I haven't
actually started studying yet!'' syndrome. When they learn you're taking PSYC
51.09, they beg you for some tips to help them quickly learn the structures.
What can you tell your friend to help them learn their amino acids quickly in
time for their exam? Should your friend study the flashcards in the same order
each time, or shuffle the flashcards with each repetition? Should they ``peek''
at the answers they're unsure about? Should they shuffle only the ``tricky''
structures (e.g., the ones they got wrong) back into the deck with each round
of repetitions, or should they go through the full set of flashcards each time?
When should they stop studying? Should they just take a nap and forget about
studying altogether? Provide four recommendations, along with clear
explanations for why you are making those particular recommendations.
\textbf{(4--8(ish) paragraphs; 1--2 per recomendation. Each recommendation must
include a clear explanation of the recommended studying technique and the
reason you are recommending it, and you should use 1--2 paragraphs to describe
and explain each recommendation.)}

\item The recently developed synthetic lifeform \textit{Roboticus Metallicus} has a
  perfect memory for everything it experiences.  It also has no way of
  prioritizing one memory over another, or of strengthening one
  association versus another.  Suppose a member of the species signs
  up to do the free association experiment you're running for your
  senior project.
\begin{enumerate}
\item Would you expect the \textit{Roboticus Metallicus}'s free
  association responses to differ from those of a typical \textit{Homo
    Sapiens}?  If so, how?  If not, why not?  \textbf{(2 paragraphs)}
\item Suppose you run many \textit{Roboticus Metallici} in your
  experiment-- enough to build up a large and reliable database of free
  associate responses.  Next you decide to construct a ``thought space''
  based on their responses (analogous to the ``Word Association
  Space'' idea we discussed in class).  Describe some features of the
  \textit{Roboticus Metallicus}-based thought space (e.g. as compared
  with Word Association Spaces constructed using human data).
  What would the thought space ``look'' like?  How would words be
  arranged?
  \textbf{(1-2 paragraphs.)}
\end{enumerate}
\end{enumerate}

\end{document}


