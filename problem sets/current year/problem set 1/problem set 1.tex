\documentclass[11pt]{article}
\usepackage{amsmath}
\usepackage{setspace}
\usepackage{pxfonts}
%\usepackage{graphicx}
\usepackage{geometry}


\geometry{letterpaper,left=.5in,right=.5in,top=0.5in,bottom=.75in,headsep=5pt,footskip=20pt}

\title{PSYC 51.09: Problem Set 1}
%\author{Jeremy R. Manning}
\date{}

\begin{document}
\maketitle
\vspace{-0.75in}
\section*{Overview}

This problem set is intended to solidify the concepts you learned about in this
week's lectures and readings. \textit{After attempting each question on your
own}, you are encouraged to work together with your classmates in small groups,
consult with ChatGPT or other tools, and/or to post and answer questions on the
course’s Canvas site.

Please upload your problem set to Canvas (as a Word or PDF file)
before the due date.  No late submissions will be accepted.

\section*{Readings and ungraded questions}

Links to the readings and materials referenced below may be found on the course GitHub page.
\begin{enumerate}
\item Read Chapter 1 of \textit{Foundations of Human Memory}.  What were your thoughts on the reading?
  \textbf{(Ungraded)}

\item Optional: read Tulving (1972)'s highly influential book chapter
  on Episodic and Semantic memory
  \textbf{(Ungraded)}

\item Optional: watch the movie \underline{Memento} (2000).  \textbf{(Ungraded)}

\item Optional: submit a multiple-choice question based on the materials
covered in this week's lectures, readings, and this problem set. You should
calibrate the difficulty so that 60--70\% of your classmates answer it
correctly on an exam.  If your question is chosen and you hit your target,
you will receive and extra credit point on that exam.  \textbf{(Ungraded)}

\end{enumerate}

\section*{Graded questions}
\begin{enumerate}

\item Reflect on your own experience with memory.  What is something that you
are reasonably certain is true about memory (based on your own subjective
experiences and intuitions), but that you don't think could be formally
studied, or that isn't generally known by the broader scientific community?
Explain your reasoning. \textbf{(2-3 paragraphs)}

\item Pretend you're an alien vistor to Earth. Your species maintains a
collective ``hive'' memory that every individual can access at will. As the
discoverer of \textit{homo sapiens}, a potentially intelligent species, you are
tasked with understanding how it thinks and remembers. Describe how you go
about studying the newly discovered species and what you conclude. It's
important that you maintain the utmost scientific rigor, since your conclusions
will be used to determine whether Earth is a good candidate for decompiling
into matter substrates for the Galaxy-Engine. You certainly wouldn't want to
wipe out another intelligence! But then again, as you know, feeding the
Galaxy-Engine with life-enriching matter substrates is what keeps your own
species alive. \textbf{(3--4 paragraphs)}

\item About a year ago, you and a close friend had a conversation that has
shaped your friendship-- and even how you think about the world at large! You
both consider that conversation to be a formative event in your lives. But,
shockingly, when you brought it up again the other day, it became clear that
each of you had very different memories about what the conversation was about,
let alone specific details of what was said. How could this be? And is there
any way you could resolve the discrepancy? (In resolving the discrepancy,
assume there were no formal records or recordings of the conversation.)
\textbf{(2-3 paragraphs)} 

\end{enumerate}

\end{document}
